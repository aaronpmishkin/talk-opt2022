%! TEX root = ../main.tex

%% Illustration of cone decomposition. 

\begin{tikzpicture}[scale=1,
		declare function={
				cv(\x) = ((abs(\x)^3) / 4 - \x^2/2 + \x/2);
				gmin(\x) = 10*(\x - 1)^2 - 2;
			}
	]
	\begin{axis}[width=1.1\linewidth, height=6cm,
			axis lines=center, yticklabels={,,}, xticklabels={,,},
			ymin=-4, ymax=4, ytick={-5,...,5}, ylabel=$$, x axis line style={-},
				xmin=-6, xmax=6, xtick={-5,...,5}, xlabel=$$, y axis line style={-},
		]
		% function
		\addplot[name path=cv_left, domain=-6:0.8, samples=100, line width=1pt]{cv(x)};
		\addplot[name path=cv_right, domain=1.2:6, samples=100, line width=1pt]{cv(x)};

		%% discontinuity
		\addplot[name path=cv_discont, domain=0.8:1.2, samples=10, line width=1pt]{gmin(x)};
		%\addplot +[mark=none, line width=1pt, dashed, draw=red] coordinates {(0.8, -1.85) (0.8, 0.22)};
		%\addplot +[mark=none, line width=1pt, dashed, draw=red] coordinates {(1.2, -1.85) (1.2, 0.322)};



		\addplot +[mark=none, line width=1pt, dashed, draw=blue] coordinates {(0.8, 0.322) (0.8, 1.36)};
		\addplot +[mark=none, line width=1pt, dashed, draw=blue] coordinates {(1.2, 0.322) (1.2, 1.36)};

		\node [circle, fill=white, inner sep=1.2pt, draw=black, line width=1pt] at (axis cs:0.8, 0.22) {};
		\node [circle, fill=white, inner sep=1.2pt, draw=black, line width=1pt] at (axis cs:1.2, 0.322) {};

		\node [circle, fill=black, inner sep=1.0pt] at (axis cs:0.8, -1.6) {};
		\node [circle, fill=black, inner sep=1.0pt] at (axis cs:1.2, -1.6) {};

		% epsilon net 
		%\addplot [name path=cv_left, domain=-4:0.5, samples=8, only marks, mark=|, fill=blue, draw=blue, mark size=5, line width=1]{cv(x)};

		%\addplot [name path=cv_left, domain=1.5:5, samples=7, only marks, mark=|, fill=blue, draw=blue, mark size=5, line width=1]{cv(x)};

		% labels

		\node[label={270:$\lambda^*$}, circle, fill=red, inner sep=1.4pt] (opt) at (axis cs:1, -2) {};

		\node[label={270:$\calV(\lambda)$}] at (axis cs:-2.4, 3) {};

		\draw [decorate, decoration = {calligraphic brace}, line width=1.2pt] (axis cs:0.8, 1.43) --  (axis cs:1.2, 1.43);

		\node[label={90:$\epsilon$}] at (axis cs:1.0, 1.41) {};
	\end{axis}

\end{tikzpicture}%
